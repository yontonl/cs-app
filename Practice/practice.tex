\documentclass[a4paper]{article}
\usepackage[margin=1in]{geometry} % 设置边距,符合Word设定
\usepackage{indentfirst}
\usepackage{fontspec}
\usepackage{listings}
\lstset{
    basicstyle=\ttfamily
}

\newcommand{\wav}{\textasciitilde}

\setmainfont{Times New Roman}

\begin{document}
\setmonofont{Courier New}
%%%%%%%%%%%%%%%%%%%%%%%%%%%%%%%%%%%%%%%%%%%%%%%%%%%%%%%%%%%%%%
% Chapter 1 practice, solution page 64.
\section*{Chapter 1}

% Page 58
\subsection*{Problem 1.1}
\begin{enumerate}
\item [A.] $25\div((1000\div100)+(1500\div150))=1.25\times$
\item [B.] $1500\div(25\div1.67-10)=300(km/h)$
\end{enumerate}

% Page 59
\subsection*{Problem 1.2}
$(1-0.1)\div(\frac{1}{4}-0.1)=6\times$

%%%%%%%%%%%%%%%%%%%%%%%%%%%%%%%%%%%%%%%%%%%%%%%%%%%%%%%%%%%%%%
% Chapter 2 practice, solution page 179.
\section*{Chapter 2}

% Page 73
\subsection*{Problem 2.1}
\begin{enumerate}
    \item [A.] \texttt{0010 0101 1011 1001 1101 0010}
    \item [B.] \texttt{0xAE49}
    \item [C.] \texttt{1010 1000 1011 0011 1101}
    \item [D.] \texttt{0x322D96}
\end{enumerate}

\subsection*{Problem 2.2}
\begin{tabular}{ccc}
    $n$&$2^n$(decimal)&$2^n$(hexadecimal)\\
    \hline
    5   &32     &\texttt{0x20}\\
    23  &8388608&\texttt{0x80000}\\
    15  &32768  &\texttt{0x8000}\\
    13  &8192   &\texttt{0x2000}\\
    12  &4096   &\texttt{0x1000}\\
    6   &64     &\texttt{0x40}\\
    8   &256    &\texttt{0x100}
\end{tabular}

% Page 74
\subsection*{Problem 2.3}
\begin{tabular}{ccc}
    Decimal &Binary      &Hexadecimal\\
    \hline
    0       &\texttt{0000 0000}   &\texttt{0x00}\\
    158     &\texttt{1001 1110}   &\texttt{0x9E}\\
    76      &\texttt{0100 1100}   &\texttt{0x7C}\\
    145     &\texttt{1001 0001}   &\texttt{0x91}\\
    174     &\texttt{1010 1110}   &\texttt{0xAE}\\
    60      &\texttt{0011 1100}   &\texttt{0x3C}\\
    241     &\texttt{1111 0001}   &\texttt{0xF1}\\
    116     &\texttt{0111 0101}   &\texttt{0x75}\\
    189     &\texttt{1011 1101}   &\texttt{0xBD}\\
    245     &\texttt{1111 0101}   &\texttt{0xF5}
\end{tabular}

% Page 75
\subsection*{Problem 2.4}
\begin{enumerate}
    \item [A.] \texttt{0x605C+0x5=0x6061}
    \item [B.] \texttt{0x605C-0x20=0x603C}
    \item [C.] \texttt{0x605C+32=0x607C}
    \item [D.] \texttt{0x60FA-0x605C=0x9e}
\end{enumerate}

% Page 84
\subsection*{Problem 2.5}
\begin{enumerate}
    \item [A.] Little endian: 78 Big endian: 12
    \item [B.] Little endian: 78 56 Big endian: 12 34
    \item [C.] Little endian: 78 56 34 Big endian: 12 34 56
\end{enumerate}

% Page 85
\subsection*{Problem 2.6}
\begin{enumerate}
    \item [A.] \texttt{0x0027C8F8: 0000 0000 0010 0111 1100 1000 1111 1000}\\
               \texttt{0x4A1F23E0: 0100 1010 0001 1111 0010 0011 1110 0000}
\end{enumerate}

% Page 85
\subsection*{Problem 2.7}
\texttt{6d 6e 6f 70 71 72}

% Page 88
\subsection*{Problem 2.8}
\begin{tabular}{cc}
    Operation                    &Result\\
    \hline
    \texttt{a}                   &\texttt{01001110}\\
    \texttt{b}                   &\texttt{11100001}\\
    \texttt{\textasciitilde a}   &\texttt{10110001}\\
    \texttt{\textasciitilde b}   &\texttt{00011110}\\
    \texttt{a\&b}                &\texttt{01000000}\\
    \texttt{a|b}                 &\texttt{11101111}\\
    \texttt{a\^{}b}              &\texttt{10101111}
\end{tabular}

% Page 89
\subsection*{Problem 2.9}
\begin{enumerate}
    \item [A.] 
        \begin{tabular}{cc}
            Color&Complement\\
            \hline
            White   &Black\\
            Blue    &Yellow\\
            Green   &Magenta\\
            Cyan    &Red\\
        \end{tabular}
    \item [B.]  Blue|Green = Cyan\\
                Yellow\&Cyan = Green\\
                Red\texttt{\^}Magenta = Blue
\end{enumerate}

% Page 90
\subsection*{Problem 2.10}
\begin{tabular}{ccc}
    Step&\tt{*x}&\tt{*y}\\
    \hline
    Initially   &\tt{a}  &\tt{b}\\
    Step 1      &\tt{a}  &\tt{a\^{}b}\\
    Step 2      &\tt{b}  &\tt{a\^{}b}\\
    Step 3      &\tt{b}  &\tt{a}
\end{tabular}

% Page 91
\subsection*{Problem 2.11}
\begin{enumerate}
    \item [A.] first=last=k
    \item [B.] The funtion implace\_swap was given two identical arguments.
    \item [C.] Change line 4 to \tt{"first < last"}.
\end{enumerate}

% Page 91
\subsection*{Problem 2.12}
\begin{enumerate}
    \item [A.] \tt{x\&0xFF}
    \item [B.] \tt{x\^{}\textasciitilde0xFF}
    \item [C.] \tt{x|0xFF}
\end{enumerate}

% Page 92
\subsection*{Problem 2.13}
\begin{itemize}
    \item[] \tt{bis(x, y)}
    \item[] \tt{bis(bic(x, y), bitc(y, x))} 
\end{itemize}

% Page 93
\subsection*{Problem 2.14}
\begin{tabular}{cccc}
    Expression  &Value      &Expression &Value\\
    \hline
    \tt{a\&b}   &\tt{0x44}  &\tt{a\&\&b}&1\\
    \tt{a|b}    &\tt{0x57}  &\tt{a||b}  &1\\
    \tt{\wav a|\wav b}&\tt{0xBB}&\tt{!a||!b}&0\\
    \tt{a\&!b}  &\tt{0}     &\tt{a\&\&\wav b}&1
    
\end{tabular}

% Page 93
\subsection*{Problem 2.15}
\begin{lstlisting}
bool equal(int x, int y) {
    return !(x ^ y);
}
\end{lstlisting}

% Page 94
\subsection*{Problem 2.16}
\begin{tabular}{cc|cc|cc|cc}
    \multicolumn{2}{c|}{\tt{a}}&\multicolumn{2}{|c|}{\tt{a<<2}}&\multicolumn{2}{|c|}{\tt{a>>3}(Logical)}&\multicolumn{2}{|c}{\tt{a>>3}(Arithmetic)}\\
    \hline
    Hex&Binary&Hex&Binary&Hex&Binary&Hex&Binary\\
    \hline
    \tt{0xD4}&\tt{1101 0100}&\tt{0x50}&\tt{0101 0000}&\tt{0x1A}&\tt{0001 1010}&\tt{0xFA}&\tt{1111 1010}\\
    \tt{0x64}&\tt{0110 0100}&\tt{0x90}&\tt{1001 0000}&\tt{0x0C}&\tt{0000 1100}&\tt{0x0C}&\tt{0000 1100}\\
    \tt{0x72}&\tt{0111 0010}&\tt{0xC8}&\tt{1100 1000}&\tt{0x0E}&\tt{0000 1110}&\tt{0x0E}&\tt{0000 1110}\\
    \tt{0x44}&\tt{0100 0100}&\tt{0x10}&\tt{0001 0000}&\tt{0x08}&\tt{0000 1000}&\tt{0x08}&\tt{0000 1000}
\end{tabular}

% Page 101
\subsection*{Problem 2.17}
\begin{tabular}{c|c|c|c}
    Hexadecimal &Binary     &$B2U_4(\vec{x})$   &$B2T_4(\vec{x})$\\
    \hline
    \tt{0xA}    &\tt{1010}  &$2^3+2^1=10$       &$-2^3+2^1=-6$\\
    \tt{0x1}    &\tt{0001}  &$2^0=1$            &$2^0=1$\\
    \tt{0xB}    &\tt{1011}  &$2^3+2^1+2^0=11$   &$-2^3+2^1+2^0=-5$\\
    \tt{0x2}    &\tt{0010}  &$2^1=2$            &$2^1=2$\\
    \tt{0x7}    &\tt{0111}  &$2^2+2^1+2^0=7$    &$2^2+2^1+2^0=7$\\
    \tt{0xC}    &\tt{1100}  &$2^3+2^2=12$       &$-2^3+2^2=-4$
\end{tabular}

% Page 105
\subsection*{Problem 2.18}
\begin{enumerate}
    \item [A.] \tt{0x2e0=736}
    \item [B.] \tt{-0x58=-88}
    \item [C.] \tt{0x28=40}
    \item [D.] \tt{-0x30=-48}
    \item [E.] \tt{0x78=120}
    \item [F.] \tt{0x88=136}
    \item [G.] \tt{0x1f8=504}
    \item [H.] \tt{0xc0=192}
    \item [I.] \tt{-0x48=-72}
\end{enumerate}

% page 107
\subsection*{Problem 2.19}
\begin{tabular}{cc}
    x&$T2U_4(x)$\\
    \hline
    -1&15\\
    -5&11\\
    -6&10\\
    -4&12\\
    1&1\\
    8&8
\end{tabular}

% Page 109
\subsection*{Problem 2.20}
pass

% Page 112
\subsection*{Problem 2.21}
\begin{tabular}{ccc}
    Expression&Type&Evaluation\\
    \hline
    \tt{-2147483647-1 == 2147483648U}&Unsigned&1\\
    \tt{-2147483647-1 < 2147483647}&Signed&1\\
    \tt{-2147483647-1U < 2147483647}&Unsigned&0\\
    \tt{-2147483647-1 < -2147483647}&Signed&1\\
    \tt{-2147483647-1U < -2147483647}&Unsigned&1
    
\end{tabular}

% Page 115
\subsection*{Problem 2.22}
\begin{enumerate}
    \item [A.] $1100_2=-2^3+2^2=-4$
    \item [B.] $11100_2=-2^4+2^3+2^2=-4$
    \item [C.] $111100_2=-2^5+2^4+2^3+2^2=-4$
\end{enumerate}

% Page 116
\subsection*{Problem 2.23}
\begin{enumerate}
    \item [A.]\begin{tabular}{c|c|c}
            w&fun1(w)&fun2(w)\\
            \hline
            \tt{0x00000076}&\tt{0x00000076}&\tt{0x00000076}\\
            \tt{0x87654321}&\tt{0x00000021}&\tt{0x00000021}\\
            \tt{0x000000C9}&\tt{0x000000C9}&\tt{0xFFFFFFC9}\\
            \tt{0xEDCBA987}&\tt{0x00000087}&\tt{0xFFFFFF87}\\
          \end{tabular}
    \item [B.] fun1 return the zero extension of the least significant byte of w.\\
               fun2 return the sign extension of the least significant byte of w.
\end{enumerate}

% Page 118
\subsection*{Problem 2.24}
\begin{tabular}{cc|cc|cc}
    \multicolumn{2}{c|}{Hex}&\multicolumn{2}{|c|}{Unsigned}&\multicolumn{2}{|c}{Signed}\\
    \hline
    Original&Truncated&Original&Truncated&Original&Truncated\\
    \hline
    1&1&1&{1}&1&{1}\\
    3&3&3&{3}&3&{3}\\
    5&5&5&{5}&5&{5}\\
    C&4&12&{4}&-4&{-4}\\
    E&6&14&{6}&-2&{-2}
\end{tabular}

% Page 119
\subsection*{Problem 2.25}
\begin{enumerate}
\item []Reason: When length equals 0, length minus 1 equals $UMax_{32}$, so the expression $i \le UMax_{32}$ holds for any unsigned $i$ and hence the for loop would never stop.
\item []Correction: Change the expression \tt{i <= length - 1} \textrm{to} \tt{i < length}.
\end{enumerate}

\subsection*{Problem 2.26}
\begin{enumerate}
    \item [A.] When string s is shorter than string t.
    \item [B.] The data type of {strlen(s)-strlen(t)} is unsigned, so it will be greater than 0 for any different strlen(s) and strlen(t). So when s is short than t, this function will return a wrong answer.
    \item [C.] Change the return value to strlen(s) > strlen(t).
\end{enumerate}

% Page 125
\subsection*{Problem 2.27}
\begin{lstlisting}
int uadd_ok(unsigned x, unsigned y) {
    return x + y < x || x + y < y;
}
\end{lstlisting}

\subsection*{Problem 2.28}
\begin{tabular}{c|c|c|c}
    \multicolumn{2}{c|}{$x$}&\multicolumn{2}{c}{$-^u_4x$}\\
    \hline
    Hex&Decimal&Decimal&Hex\\
    \hline
    1&1&15&F\\
    4&4&11&B\\
    7&7&9&9\\
    A&10&6&6\\
    E&14&2&2\\

\end{tabular}

% Page 129
\subsection*{Problem 2.29}
\begin{tabular}{ccccc}
    $x$&$y$&$x+y$&$x+^t_5y$&Case\\
    \hline
    -12&-15&-27&5&1\\
    10100&10001&100101&00101&1\\
    -8&-8&-16&-16&2\\
    11000&11000&110000&10000&2\\
    -9&8&-1&-1&2\\
    10111&01000&11111&11111&2\\
    2&5&7&7&3\\
    00010&00101&00111&00111&3\\
    12&4&16&-16&4\\
    01100&00100&10000&10000&4
\end{tabular}

% Page 130
\subsection*{Problem 2.30}
\begin{lstlisting}
int tadd_ok(int x, int y) {
    int z = x + y;
    return !((x > 0 && y > 0 && z <= 0) || (x < 0 && y < 0 && z >= 0));
}
\end{lstlisting}

\subsection*{Problem 2.31}
Signed addition is associative and commutative, so (x+y)-x = y+(x-x)=y and hence whether or not there is an overflow, this function will always return 1.

\subsection*{Problem 2.32}
For any x and y = $TMin$, this function will give incorrect results.

% Page 131
\subsection*{Problem 2.33}
\begin{tabular}{c|c|c|c}
    \multicolumn{2}{c|}{$x$}&\multicolumn{2}{c}{$-^t_4x$}\\
    \hline
    Hex&Decimal&Decimal&Hex\\
    \hline
    2&2&-2&E\\
    3&3&-3&D\\
    9&9&-9&7\\
    B&11&-11&5\\
    C&12&-12&4\\

\end{tabular}\\
The bit patterns generated by two's complement and unsigned negation are identical.

% Page 134
\subsection*{Problem 2.34}
\begin{tabular}{c|cc|cc|cc|cc}
    Mode&\multicolumn{2}{|c|}{x}&\multicolumn{2}{|c|}{y}&\multicolumn{2}{|c|}{$x\cdot y$}&\multicolumn{2}{c}{Truncated $x\cdot y$}\\
    \hline
    {}&Hex&Binary&Hex&Binary&Hex&Binary&Hex&Binary\\
    \hline
    Unsigned&4&100&5&101&20&010100&4&100\\
    Two's complement&-4&100&-3&101&12&001100&-4&100\\
    \hline
    Unsigned&2&010&7&111&14&001110&6&110\\
    Two's complement&2&010&-1&111&-2&111110&-2&110\\
    \hline
    Unsigned&6&110&6&110&36&100100&4&110\\
    Two's complement&-2&110&-2&110&4&000100&-4&100
\end{tabular}

% Page 135
\subsection*{Problem 2.35}
pass

\subsection*{Problem 2.36}
\begin{lstlisting}
int tmult_ok(int x, int y) {
    int64_t z1 = (int64_t)x * y;
    int z2 = x * y;
    return (int64_t)z2 == z1;
}
\end{lstlisting}

\subsection*{Problem 2.37}
\begin{enumerate}
    \item [A.] No improvement at all. Although variable asize is 64-bit and its value is accurate, when it is passed to malloc as a parameter with type size\_t, it will still be truncated to 32 bit as well.
    \item [B.] Since the parameter of malloc is size\_t with 32 bit, it's impossible to allocate more than $2^{32}$ bytes. What we can do is to determine whether there is an overflow before malloc. If there is, do not call malloc and return NULL.
\end{enumerate}

% Page 138
\subsection*{Problem 2.38}
A power of 2$(2^k, for\ any\ k > 0)$ or A power of 2 plus 1$(2^k+1, for\ any\ k > 0)$.

% Page 139
\subsection*{Problem 2.39}
$-(x<<m)$

\subsection*{Problem 2.40}
\begin{tabular}{cccc}
    K&Shifts&Add/Subs&Expression\\
    \hline
    7&1&1&$(x<<3)-x$\\
    30&4&3&$(x<<4)+(x<<3)+(x<<2)+(x<<1)$\\
    28&2&1&$(x<<5)-(x<<2)$\\
    55&2&2&$(x<<6)-(x<<3)-x$\\
\end{tabular}

\subsection*{Problem 2.41}
When m = n and m + 1 = n, choose form A, otherwise form B.

% Page 143
\subsection*{Problem 2.42}
\begin{lstlisting}
int div16(int x) {
    return (x + ((x >> 31) & 0xF)) >> 4;
}
\end{lstlisting}

\subsection*{Problem 2.43}
M = 31, N = 8.

% Page 144
\subsection*{Problem 2.44}
\begin{enumerate}
    \item [A.] False for x = -2147483648.
    \item [B.] True. If (x \& 7) != 7 is false, namely (x 
    \& 7) == 7, the least 3 significant bits must be [111]. So the most 3 significant bits of x << 29 will be 111 and hence x << 29 < 0.
    \item [C.] False for x = 50000 where the value of x * x equals 2500000000 > 2147483647, causes positive overflow and yields a negative value.
    \item [D.] True. For any x >= 0, -x must be smaller than or equal to 0. Negation of a nonnegative value will never cause an overflow.
    \item [E.] False. This is true for any value of type \tt{int} \rm{except} -2147483648. Negation of -2147483648 is still -2147483648 and is smaller than 0.
    \item [F.] True. Two's complement addition has the same bit-level represetation as unsigned.
    \item [G.] True
\end{enumerate}


\end{document}