\documentclass[a4paper]{article}
\usepackage[margin=1in]{geometry} % 设置边距,符合Word设定
\usepackage{indentfirst}
\usepackage{fontspec}
\usepackage{listings}
\usepackage{color}
\lstset{
    basicstyle=\ttfamily\small,
    language=C
}

\newcommand{\wav}{\textasciitilde}

\setmainfont{Times New Roman}
\setmonofont{Courier}

\begin{document}

%%%%%%%%%%%%%%%%%%%%%%%%%%%%%%%%%%%%%%%%%%%%%%%%%%%%%%%%%%%%%%

\section*{Chapter 1}

\subsection*{Homework Problems}

% Page 58
\subsubsection*{Problem 1.1}
\begin{enumerate}
\item [A.] $25\div((1000\div100)+(1500\div150))=1.25\times$
\item [B.] $1500\div(25\div1.67-10)=300(km/h)$
\end{enumerate}

% Page 59
\subsubsection*{Problem 1.2}
$(1-0.1)\div(\frac{1}{4}-0.1)=6\times$

%%%%%%%%%%%%%%%%%%%%%%%%%%%%%%%%%%%%%%%%%%%%%%%%%%%%%%%%%%%%%%

\section*{Chapter 2}

\subsection*{Practice Problems}

% Page 73
\subsubsection*{Problem 2.1}
\begin{enumerate}
    \item [A.] \texttt{0010 0101 1011 1001 1101 0010}
    \item [B.] \texttt{0xAE49}
    \item [C.] \texttt{1010 1000 1011 0011 1101}
    \item [D.] \texttt{0x322D96}
\end{enumerate}

\subsubsection*{Problem 2.2}
\begin{tabular}{ccc}
    $n$&$2^n$(decimal)&$2^n$(hexadecimal)\\
    \hline
    5   &32     &\texttt{0x20}\\
    23  &8388608&\texttt{0x80000}\\
    15  &32768  &\texttt{0x8000}\\
    13  &8192   &\texttt{0x2000}\\
    12  &4096   &\texttt{0x1000}\\
    6   &64     &\texttt{0x40}\\
    8   &256    &\texttt{0x100}
\end{tabular}

% Page 74
\subsubsection*{Problem 2.3}
\begin{tabular}{ccc}
    Decimal &Binary      &Hexadecimal\\
    \hline
    0       &\texttt{0000 0000}   &\texttt{0x00}\\
    158     &\texttt{1001 1110}   &\texttt{0x9E}\\
    76      &\texttt{0100 1100}   &\texttt{0x7C}\\
    145     &\texttt{1001 0001}   &\texttt{0x91}\\
    174     &\texttt{1010 1110}   &\texttt{0xAE}\\
    60      &\texttt{0011 1100}   &\texttt{0x3C}\\
    241     &\texttt{1111 0001}   &\texttt{0xF1}\\
    116     &\texttt{0111 0101}   &\texttt{0x75}\\
    189     &\texttt{1011 1101}   &\texttt{0xBD}\\
    245     &\texttt{1111 0101}   &\texttt{0xF5}
\end{tabular}

% Page 75
\subsubsection*{Problem 2.4}
\begin{enumerate}
    \item [A.] \texttt{0x605C+0x5=0x6061}
    \item [B.] \texttt{0x605C-0x20=0x603C}
    \item [C.] \texttt{0x605C+32=0x607C}
    \item [D.] \texttt{0x60FA-0x605C=0x9e}
\end{enumerate}

% Page 84
\subsubsection*{Problem 2.5}
\begin{enumerate}
    \item [A.] Little endian: 78 Big endian: 12
    \item [B.] Little endian: 78 56 Big endian: 12 34
    \item [C.] Little endian: 78 56 34 Big endian: 12 34 56
\end{enumerate}

% Page 85
\subsubsection*{Problem 2.6}
\begin{enumerate}
    \item [A.] \texttt{0x0027C8F8: 0000 0000 0010 0111 1100 1000 1111 1000}\\
               \texttt{0x4A1F23E0: 0100 1010 0001 1111 0010 0011 1110 0000}
\end{enumerate}

% Page 85
\subsubsection*{Problem 2.7}
\texttt{6d 6e 6f 70 71 72}

% Page 88
\subsubsection*{Problem 2.8}
\begin{tabular}{cc}
    Operation                    &Result\\
    \hline
    \texttt{a}                   &\texttt{01001110}\\
    \texttt{b}                   &\texttt{11100001}\\
    \texttt{\textasciitilde a}   &\texttt{10110001}\\
    \texttt{\textasciitilde b}   &\texttt{00011110}\\
    \texttt{a\&b}                &\texttt{01000000}\\
    \texttt{a|b}                 &\texttt{11101111}\\
    \texttt{a\^{}b}              &\texttt{10101111}
\end{tabular}

% Page 89
\subsubsection*{Problem 2.9}
\begin{enumerate}
    \item [A.] 
        \begin{tabular}{cc}
            Color&Complement\\
            \hline
            White   &Black\\
            Blue    &Yellow\\
            Green   &Magenta\\
            Cyan    &Red\\
        \end{tabular}
    \item [B.]  Blue|Green = Cyan\\
                Yellow\&Cyan = Green\\
                Red\texttt{\^}Magenta = Blue
\end{enumerate}

% Page 90
\subsubsection*{Problem 2.10}
\begin{tabular}{ccc}
    Step&\tt{*x}&\tt{*y}\\
    \hline
    Initially   &\tt{a}  &\tt{b}\\
    Step 1      &\tt{a}  &\tt{a\^{}b}\\
    Step 2      &\tt{b}  &\tt{a\^{}b}\\
    Step 3      &\tt{b}  &\tt{a}
\end{tabular}

% Page 91
\subsubsection*{Problem 2.11}
\begin{enumerate}
    \item [A.] first=last=k
    \item [B.] The funtion implace\_swap was given two identical arguments.
    \item [C.] Change line 4 to \tt{"first < last"}.
\end{enumerate}

% Page 91
\subsubsection*{Problem 2.12}
\begin{enumerate}
    \item [A.] \tt{x\&0xFF}
    \item [B.] \tt{x\^{}\textasciitilde0xFF}
    \item [C.] \tt{x|0xFF}
\end{enumerate}

% Page 92
\subsubsection*{Problem 2.13}
\begin{itemize}
    \item[] \tt{bis(x, y)}
    \item[] \tt{bis(bic(x, y), bitc(y, x))} 
\end{itemize}

% Page 93
\subsubsection*{Problem 2.14}
\begin{tabular}{cccc}
    Expression  &Value      &Expression &Value\\
    \hline
    \tt{a\&b}   &\tt{0x44}  &\tt{a\&\&b}&1\\
    \tt{a|b}    &\tt{0x57}  &\tt{a||b}  &1\\
    \tt{\wav a|\wav b}&\tt{0xBB}&\tt{!a||!b}&0\\
    \tt{a\&!b}  &\tt{0}     &\tt{a\&\&\wav b}&1
    
\end{tabular}

% Page 93
\subsubsection*{Problem 2.15}
\begin{lstlisting}
bool equal(int x, int y) {
    return !(x ^ y);
}
\end{lstlisting}

% Page 94
\subsubsection*{Problem 2.16}
\begin{tabular}{cc|cc|cc|cc}
    \multicolumn{2}{c|}{\tt{a}}&\multicolumn{2}{|c|}{\tt{a<<2}}&\multicolumn{2}{|c|}{\tt{a>>3}(Logical)}&\multicolumn{2}{|c}{\tt{a>>3}(Arithmetic)}\\
    \hline
    Hex&Binary&Hex&Binary&Hex&Binary&Hex&Binary\\
    \hline
    \tt{0xD4}&\tt{1101 0100}&\tt{0x50}&\tt{0101 0000}&\tt{0x1A}&\tt{0001 1010}&\tt{0xFA}&\tt{1111 1010}\\
    \tt{0x64}&\tt{0110 0100}&\tt{0x90}&\tt{1001 0000}&\tt{0x0C}&\tt{0000 1100}&\tt{0x0C}&\tt{0000 1100}\\
    \tt{0x72}&\tt{0111 0010}&\tt{0xC8}&\tt{1100 1000}&\tt{0x0E}&\tt{0000 1110}&\tt{0x0E}&\tt{0000 1110}\\
    \tt{0x44}&\tt{0100 0100}&\tt{0x10}&\tt{0001 0000}&\tt{0x08}&\tt{0000 1000}&\tt{0x08}&\tt{0000 1000}
\end{tabular}

% Page 101
\subsubsection*{Problem 2.17}
\begin{tabular}{c|c|c|c}
    Hexadecimal &Binary     &$B2U_4(\vec{x})$   &$B2T_4(\vec{x})$\\
    \hline
    \tt{0xA}    &\tt{1010}  &$2^3+2^1=10$       &$-2^3+2^1=-6$\\
    \tt{0x1}    &\tt{0001}  &$2^0=1$            &$2^0=1$\\
    \tt{0xB}    &\tt{1011}  &$2^3+2^1+2^0=11$   &$-2^3+2^1+2^0=-5$\\
    \tt{0x2}    &\tt{0010}  &$2^1=2$            &$2^1=2$\\
    \tt{0x7}    &\tt{0111}  &$2^2+2^1+2^0=7$    &$2^2+2^1+2^0=7$\\
    \tt{0xC}    &\tt{1100}  &$2^3+2^2=12$       &$-2^3+2^2=-4$
\end{tabular}

% Page 105
\subsubsection*{Problem 2.18}
\begin{enumerate}
    \item [A.] \tt{0x2e0=736}
    \item [B.] \tt{-0x58=-88}
    \item [C.] \tt{0x28=40}
    \item [D.] \tt{-0x30=-48}
    \item [E.] \tt{0x78=120}
    \item [F.] \tt{0x88=136}
    \item [G.] \tt{0x1f8=504}
    \item [H.] \tt{0xc0=192}
    \item [I.] \tt{-0x48=-72}
\end{enumerate}

% page 107
\subsubsection*{Problem 2.19}
\begin{tabular}{cc}
    x&$T2U_4(x)$\\
    \hline
    -1&15\\
    -5&11\\
    -6&10\\
    -4&12\\
    1&1\\
    8&8
\end{tabular}

% Page 109
\subsubsection*{Problem 2.20}
pass

% Page 112
\subsubsection*{Problem 2.21}
\begin{tabular}{ccc}
    Expression&Type&Evaluation\\
    \hline
    \tt{-2147483647-1 == 2147483648U}&Unsigned&1\\
    \tt{-2147483647-1 < 2147483647}&Signed&1\\
    \tt{-2147483647-1U < 2147483647}&Unsigned&0\\
    \tt{-2147483647-1 < -2147483647}&Signed&1\\
    \tt{-2147483647-1U < -2147483647}&Unsigned&1
    
\end{tabular}

% Page 115
\subsubsection*{Problem 2.22}
\begin{enumerate}
    \item [A.] $1100_2=-2^3+2^2=-4$
    \item [B.] $11100_2=-2^4+2^3+2^2=-4$
    \item [C.] $111100_2=-2^5+2^4+2^3+2^2=-4$
\end{enumerate}

% Page 116
\subsubsection*{Problem 2.23}
\begin{enumerate}
    \item [A.]\begin{tabular}{c|c|c}
            w&fun1(w)&fun2(w)\\
            \hline
            \tt{0x00000076}&\tt{0x00000076}&\tt{0x00000076}\\
            \tt{0x87654321}&\tt{0x00000021}&\tt{0x00000021}\\
            \tt{0x000000C9}&\tt{0x000000C9}&\tt{0xFFFFFFC9}\\
            \tt{0xEDCBA987}&\tt{0x00000087}&\tt{0xFFFFFF87}\\
          \end{tabular}
    \item [B.] fun1 return the zero extension of the least significant byte of w.\\
               fun2 return the sign extension of the least significant byte of w.
\end{enumerate}

% Page 118
\subsubsection*{Problem 2.24}
\begin{tabular}{cc|cc|cc}
    \multicolumn{2}{c|}{Hex}&\multicolumn{2}{|c|}{Unsigned}&\multicolumn{2}{|c}{Signed}\\
    \hline
    Original&Truncated&Original&Truncated&Original&Truncated\\
    \hline
    1&1&1&{1}&1&{1}\\
    3&3&3&{3}&3&{3}\\
    5&5&5&{5}&5&{5}\\
    C&4&12&{4}&-4&{-4}\\
    E&6&14&{6}&-2&{-2}
\end{tabular}

% Page 119
\subsubsection*{Problem 2.25}
\begin{enumerate}
\item []Reason: When length equals 0, length minus 1 equals $UMax_{32}$, so the expression $i \le UMax_{32}$ holds for any unsigned $i$ and hence the for loop would never stop.
\item []Correction: Change the expression \tt{i <= length - 1} \textrm{to} \tt{i < length}.
\end{enumerate}

\subsubsection*{Problem 2.26}
\begin{enumerate}
    \item [A.] When string s is shorter than string t.
    \item [B.] The data type of {strlen(s)-strlen(t)} is unsigned, so it will be greater than 0 for any different strlen(s) and strlen(t). So when s is short than t, this function will return a wrong answer.
    \item [C.] Change the return value to strlen(s) > strlen(t).
\end{enumerate}

% Page 125
\subsubsection*{Problem 2.27}
\begin{lstlisting}
int uadd_ok(unsigned x, unsigned y) {
    return x + y < x || x + y < y;
}
\end{lstlisting}

\subsubsection*{Problem 2.28}
\begin{tabular}{c|c|c|c}
    \multicolumn{2}{c|}{$x$}&\multicolumn{2}{c}{$-^u_4x$}\\
    \hline
    Hex&Decimal&Decimal&Hex\\
    \hline
    1&1&15&F\\
    4&4&11&B\\
    7&7&9&9\\
    A&10&6&6\\
    E&14&2&2\\

\end{tabular}

% Page 129
\subsubsection*{Problem 2.29}
\begin{tabular}{ccccc}
    $x$&$y$&$x+y$&$x+^t_5y$&Case\\
    \hline
    -12&-15&-27&5&1\\
    10100&10001&100101&00101&1\\
    -8&-8&-16&-16&2\\
    11000&11000&110000&10000&2\\
    -9&8&-1&-1&2\\
    10111&01000&11111&11111&2\\
    2&5&7&7&3\\
    00010&00101&00111&00111&3\\
    12&4&16&-16&4\\
    01100&00100&10000&10000&4
\end{tabular}

% Page 130
\subsubsection*{Problem 2.30}
\begin{lstlisting}
int tadd_ok(int x, int y) {
    int z = x + y;
    return !((x > 0 && y > 0 && z <= 0) || (x < 0 && y < 0 && z >= 0));
}
\end{lstlisting}

\subsubsection*{Problem 2.31}
Signed addition is associative and commutative, so (x+y)-x = y+(x-x)=y and hence whether or not there is overflow, this function will always return 1.

\subsubsection*{Problem 2.32}
For any x and y = $TMin$, this function will give incorrect results.

% Page 131
\subsubsection*{Problem 2.33}
\begin{tabular}{c|c|c|c}
    \multicolumn{2}{c|}{$x$}&\multicolumn{2}{c}{$-^t_4x$}\\
    \hline
    Hex&Decimal&Decimal&Hex\\
    \hline
    2&2&-2&E\\
    3&3&-3&D\\
    9&9&-9&7\\
    B&11&-11&5\\
    C&12&-12&4\\

\end{tabular}\\
The bit patterns generated by two's complement and unsigned negation are identical.

% Page 134
\subsubsection*{Problem 2.34}
\begin{tabular}{c|cc|cc|cc|cc}
    Mode&\multicolumn{2}{|c|}{x}&\multicolumn{2}{|c|}{y}&\multicolumn{2}{|c|}{$x\cdot y$}&\multicolumn{2}{c}{Truncated $x\cdot y$}\\
    \hline
    {}&Hex&Binary&Hex&Binary&Hex&Binary&Hex&Binary\\
    \hline
    Unsigned&4&100&5&101&20&010100&4&100\\
    Two's complement&-4&100&-3&101&12&001100&-4&100\\
    \hline
    Unsigned&2&010&7&111&14&001110&6&110\\
    Two's complement&2&010&-1&111&-2&111110&-2&110\\
    \hline
    Unsigned&6&110&6&110&36&100100&4&110\\
    Two's complement&-2&110&-2&110&4&000100&-4&100
\end{tabular}

% Page 135
\subsubsection*{Problem 2.35}
pass

\subsubsection*{Problem 2.36}
\begin{lstlisting}
int tmult_ok(int x, int y) {
    int64_t z1 = (int64_t)x * y;
    int z2 = x * y;
    return (int64_t)z2 == z1;
}
\end{lstlisting}

\subsubsection*{Problem 2.37}
\begin{enumerate}
    \item [A.] No improvement at all. Although variable asize is 64-bit and its value is accurate, when it is passed to malloc as a parameter with type size\_t, it will still be truncated to 32 bit as well.
    \item [B.] Since the parameter of malloc is size\_t with 32 bit, it's impossible to allocate more than $2^{32}$ bytes. What we can do is to determine whether there is overflow before malloc. If there is, do not call malloc and return NULL.
\end{enumerate}

% Page 138
\subsubsection*{Problem 2.38}
A power of 2$(2^k, for\ any\ k > 0)$ or A power of 2 plus 1$(2^k+1, for\ any\ k > 0)$.

% Page 139
\subsubsection*{Problem 2.39}
$-(x<<m)$

\subsubsection*{Problem 2.40}
\begin{tabular}{cccc}
    K&Shifts&Add/Subs&Expression\\
    \hline
    7&1&1&$(x<<3)-x$\\
    30&4&3&$(x<<4)+(x<<3)+(x<<2)+(x<<1)$\\
    28&2&1&$(x<<5)-(x<<2)$\\
    55&2&2&$(x<<6)-(x<<3)-x$\\
\end{tabular}

\subsubsection*{Problem 2.41}
When m = n and m + 1 = n, choose form A, otherwise form B.

% Page 143
\subsubsection*{Problem 2.42}
\begin{lstlisting}
int div16(int x) {
    return (x + ((x >> 31) & 0xF)) >> 4;
}
\end{lstlisting}

\subsubsection*{Problem 2.43}
M = 31, N = 8.

% Page 144
\subsubsection*{Problem 2.44}
\begin{enumerate}
    \item [A.] False for x = -2147483648.
    \item [B.] True. If (x \& 7) != 7 is false, namely (x 
    \& 7) == 7, the least 3 significant bits must be [111]. So the most 3 significant bits of x << 29 will be 111 and hence x << 29 < 0.
    \item [C.] False for x = 50000 where the value of x * x equals 2500000000 > 2147483647, causes positive overflow and yields a negative value.
    \item [D.] True. For any x >= 0, -x must be smaller than or equal to 0. Negation of a nonnegative value will never cause overflow.
    \item [E.] False. This is true for any value of type \tt{int} \rm{except} -2147483648. Negation of -2147483648 is still -2147483648 and is smaller than 0.
    \item [F.] True. Two's complement addition has the same bit-level represetation as unsigned.
    \item [G.] True
\end{enumerate}

% Page 147
\subsubsection*{Problem 2.45}
\begin{tabular}{c|c|c}
    Fractional value&Binary represetation&Decimal represetation\\
    \hline
    $\frac{1}{8}$&0.001&0.125\\
    \hline
    $\frac{3}{4}$&0.11&0.75\\
    \hline
    $\frac{5}{16}$&0.0101&0.3125\\
    \hline
    $2\frac{11}{16}$&10.1011&2.6875\\
    \hline
    $1\frac{1}{8}$&1.001&1.125\\
    \hline
    $5\frac{7}{8}$&101.111&5.875\\
    \hline
    $3\frac{3}{16}$&11.0011&3.1875\\
\end{tabular}

\subsubsection*{Problem 2.46}
\begin{enumerate}
    \item [A.] $0.00000000000000000000000[0011]\cdots_2$
    \item [B.] $1/(2^{20}\times10)$
    \item [C.] $(3600*100*10)/(2^{20}\times10)=0.343s$
    \item [D.] $2000m/s \times 0.343s = 687m$
\end{enumerate}

% Page 153
\subsubsection*{Problem 2.47}
\begin{tabular}{ccccccccc}
    Bits&$e$&$E$&$2^E$&$f$&$M$&$2^E\times M$&$V$&Decimal\\\hline
    \tt{0 00 00}&0&0&$\frac{1}{1}$&$\frac{0}{4}$&$\frac{0}{4}$&$\frac{0}{4}$&$0$&0.0\\\hline
    \tt{0 00 01}&0&0&$\frac{1}{1}$&$\frac{1}{4}$&$\frac{1}{4}$&$\frac{1}{4}$&$\frac{1}{4}$&0.25\\\hline
    \tt{0 00 10}&0&0&$\frac{1}{1}$&$\frac{2}{4}$&$\frac{2}{4}$&$\frac{2}{4}$&$\frac{1}{2}$&0.5\\\hline
    \tt{0 00 11}&0&0&$\frac{1}{1}$&$\frac{1}{4}$&$\frac{3}{4}$&$\frac{3}{4}$&$\frac{3}{4}$&0.75\\\hline
    \tt{0 01 00}&1&0&$\frac{1}{1}$&$\frac{0}{4}$&$\frac{4}{4}$&$\frac{4}{4}$&$1$&1.0\\\hline
    \tt{0 01 01}&1&0&$\frac{1}{1}$&$\frac{1}{4}$&$\frac{5}{4}$&$\frac{5}{4}$&$\frac{5}{4}$&1.25\\\hline
    \tt{0 01 10}&1&0&$\frac{1}{1}$&$\frac{2}{4}$&$\frac{6}{4}$&$\frac{6}{4}$&$\frac{3}{2}$&1.5\\\hline
    \tt{0 01 11}&1&0&$\frac{1}{1}$&$\frac{3}{4}$&$\frac{7}{4}$&$\frac{7}{4}$&$\frac{7}{4}$&1.75\\\hline
    \tt{0 10 00}&2&1&$2$&$\frac{0}{4}$&$\frac{4}{4}$&$\frac{8}{4}$&$2$&2.0\\\hline
    \tt{0 10 01}&2&1&$2$&$\frac{1}{4}$&$\frac{5}{4}$&$\frac{10}{4}$&$\frac{5}{2}$&2.5\\\hline
    \tt{0 10 10}&2&1&$2$&$\frac{2}{4}$&$\frac{6}{4}$&$\frac{12}{4}$&$3$&3.0\\\hline
    \tt{0 10 11}&2&1&$2$&$\frac{3}{4}$&$\frac{7}{4}$&$\frac{14}{4}$&$\frac{7}{2}$&3.5\\\hline
    \tt{0 11 00}&&&&&&&$\infty$&\\\hline
    \tt{0 11 01}&&&&&&&NaN&\\\hline
    \tt{0 11 10}&&&&&&&NaN&\\\hline
    \tt{0 11 11}&&&&&&&NaN&\\\hline

\end{tabular}

% Page 155
\subsubsection*{Problem 2.48}
\begin{enumerate}
    \item [Step 1.] From hexadecimal represetation 0x00359141, get the binary represetation: [0000 0000 0011 0101 1001 0001 0100 0001]
    \item [Step 2.] Create normalized represetation by shift 21 positions to the right of binary point, giving 1.1010 1100 1000 1010 0000 1 $\times 2^{21}$
    \item [Step 3.] Remove leading 1 of significand and add 2 zeros to the right, giving the fraction field [1010 1100 1000 1010 0000 100]
    \item [Step 4.] Since $E = 21$, exponent field $exp = E + Bias = 21 + 127 = 148$, giving the exponent field [1001 0100]
    \item [Step 5.] So the floating-point represetation of integer 3,510,593 is [0100 1010 0101 0110 0100 0101 0000 0100], 
    of which the hexadecimal represetation is \tt{0x4A564504}
\end{enumerate}

% Page 156
\subsubsection*{Problem 2.49}
\begin{enumerate}
    \item [A.] $2^{n+1}+1$
    \item [B.] 16777217
\end{enumerate}

% Page 157
\subsubsection*{Problem 2.50}
\begin{tabular}{c|c|c|c}
    \multicolumn{2}{c|}{Before}&\multicolumn{2}{c}{After}\\\hline
    Binary&Numeric value&Binary&Numeric value\\\hline
    10.111&2.875&11.0&3\\
    11.010&3.25&11.0&3\\
    11.000&3&11.0&3\\
    10.110&2.75&11.0&3
\end{tabular}

% Page 158
\subsubsection*{Problem 2.51}
\begin{enumerate}
    \item [A.] 0.00011001100110011001101
    \item [B.] $0.1\times 2^{-22}$
    \item [C.] $(3600*100*10/(10*2^{22})) = 0.0858s$
    \item [D.] $0.0858s\times2000m/s=171m$
\end{enumerate}

% Page 158
\subsubsection*{Problem 2.52}
\begin{tabular}{c|c|c|c}
    \multicolumn{2}{c|}{Format A}&\multicolumn{2}{c}{Format B}\\
    \hline
    Bits&Value&Bits&Value\\
    \hline
    011 0000&1&0111 000&1\\
    101 1110&7.5&1001 111&7.5\\
    010 1001&0.78125&0110 100&0.75\\
    110 1111&15.5&1011 000&16\\
    000 0001&$\frac{1}{2^{6}}$&0001 000&$\frac{1}{2^{6}}$ 
    
\end{tabular}

% Page 161
\subsubsection*{Problem 2.53}
\begin{lstlisting}
#define POS_INFINITY 1e309
#define NEG_INFINITY -1e309
#define NEG_ZERO -1e324
\end{lstlisting}

% Page 161
\subsubsection*{Problem 2.54}
\begin{enumerate}
    \item [A.] True, since casting an int variable to double will cause neither overflow or precision lost.
    \item [B.] False. When casting int to float, the numeric value might be rounded, such as \tt{0x7fffffff}\rm{.}
    \item [C.] False. Casting double to float might overflow and produce INFINITY, such as float 1e40.
    \item [D.] True, just like (A.)
    \item [E.] True. Negation only changes the sign bit, so there is neither overflow or rounding.
    \item [F.] True. The lefthand and righthand are both equal to 0.5.
    \item [G.] True. The sign bit of multiplication result is 0 and hence the result is nonnegative if and only if sign bits of two operands are identical.
    \item [H.] False. f+d may overflow and produce INFINITY, such as f = 10.0 and d = \{the max value of \tt{double}\rm{\}.}
\end{enumerate}

\subsection*{Homework Problems}
% Page 164
\subsubsection*{Problem 2.55}
\begin{itemize}
    \item Little endian on macOS x86-64.
    \item Little endian on Ubuntu 16.04 x86-64.
\end{itemize}

\subsubsection*{Problem 2.56}
pass

\subsubsection*{Problem 2.57}
\begin{lstlisting}
void show_short(short sx) {
    show_bytes((byte_pointer) &x, sizeof(short));
}

void show_long(long lx) {
    show_bytes((byte_pointer) &x, sizeof(long));
}

void show_double(double d) {
    show_bytes((byte_pointer) &x, sizeof(double));
}
\end{lstlisting}

\subsubsection*{Problem 2.58}
\begin{lstlisting}
int is_little_endian() {
    int32_t x = 0x12345678;
    return *((int8_t *)&x) == 0x78;
}
\end{lstlisting}

\subsubsection*{Problem 2.59}
\tt{(x \& 0xff) | (y \& \wav0xff)}

\subsubsection*{Problem 2.60}
\begin{lstlisting}
unsigned replace_byte(unsigned x, int i, unsigned char b) {
    return (x & ~(0xff << (8*i))) | (b << (8*i));
}
\end{lstlisting}

% Page 165
\subsubsection*{}



%%%%%%%%%%%%%%%%%%%%%%%%%%%%%%%%%%%%%%%%%%%%%%%%%%%%%%%%%%%%%%

\section*{Chapter 3}

\subsection*{Practice Problems}

% Page 218
\subsubsection*{Problem 3.1}
\begin{tabular}{cc}
    \rm{Operand}&\rm{Value}\\\hline
    \%\tt{rax}&0x100\\
    0x104&0xAB\\
    \$0x108&0x108\\
    (\%rax)&0xFF\\
    4(\%rax)&0xAB\\
    9(\%rax, \%rdx)&0x11\\
    260(\%rcx, \%rdx)&0x13\\
    0xFC(, \%rcx, 4)&0xFF\\
    (\%rax, \%rdx, 4)&0x11    
\end{tabular} 

% Page 221
\subsubsection*{Problem 3.2}
\begin{tabular}{ccc}
    \tt{movl}&\%eax&(\%rsp)\\
    movw&(\%rax)&\%dx\\
    movb&0xFF&\%bl\\
    movb&(\%rsp, \%rdx, 4)&\%dl\\
    movq&(\%rdx)&\%rax\\
    movw&\%dx&(\%rax)
    
\end{tabular}

% Page 222
\subsubsection*{Problem 3.3}
\begin{enumerate}
    \rm{}
    \item \%ebx is a 32-bit register, but memory address should be a 64-bit integer.
    \item \%rax is a 64-bit register, thus it cannot be source of \tt{movl}\rm{.}
    \item Oprands cannot both be memory references.
    \item There is no register named \%sl.
    \item Immediate cannot be destination.
    \item \textcolor{red}{The destination oprand of the question is not consistent with which of the solution.}
    \item \%si is a 16-bit register, thus it cannot be source of \tt{movb}\rm{.}
\end{enumerate}

% Page 223
\subsubsection*{Problem 3.4}
\begin{lstlisting}
    movq (%rdi), %rax
    movq %rax, (%rsi)
\end{lstlisting}

% Page 225
\subsubsection*{Problem 3.5}
\begin{lstlisting}
void decode(long *xp, long *yp, long *zp) {
    int t1 = *xp, t2 = *yp, t3 = *zp;
    *yp = t1;
    *zp = t2;
    *xp = t3;
    return t3;
}
\end{lstlisting}

% Page 228
\subsubsection*{Problem 3.6}
\begin{tabular}{ll}
    Instruction&Result\\\hline
    \tt{leaq 9(\%rdx), \%rax}&q+9\\
    leaq (\%rdx, \%rbx), \%rax&p+q\\
    leaq (\%rdx, \%rbx, 3), \%rax&q+3p\\
    leaq 2(\%rbx, \%rbx, 7), \%rax&8p+2\\
    leaq 0xE(, \%rdx, 3), \%rax&3q+14\\
    leaq 6(\%rbx, \%rdx, 7), \%rax&p+7q+6
\end{tabular}

% Page 229
\subsubsection*{Problem 3.7}
\tt{10*y+z+x*y}

% Page 230
\subsubsection*{Problem 3.8}
\begin{tabular}{lcl}
    Instruction&Destination&Value\\\hline
    \tt{addq \%rcx, (\%rax)}&0x100&0x100\\
    subq \%rdx, 8(\%rax)&0x108&0xA9\\
    imulq \$16, (\%rax, \%rdx, 8)&0x118&0x110\\
    incq 16(\%rax)&0x110&0x14\\
    decq \%rcx&\%rcx&0x0\\
    subq \%rdx, \%rax&\%rax&0xFD
    
\end{tabular}

% Page 231
\subsubsection*{Problem 3.9}
\begin{lstlisting}
salq $4, %rax
sarq %cl, %rax
\end{lstlisting}

% Page 232
\subsubsection*{Problem 3.10}
\begin{lstlisting}
short p1 = y|z;
short p2 = p1 >> 9;
short p3 = ~p2;
short p4 = y - p3;
\end{lstlisting}

% Page 233
\subsubsection*{Problem 3.11}
\begin{enumerate}
    \rm{}
    \item[A.] Clear register \%rcx.
    \item[B.] \tt{movq \$0, \%rcx}\rm{}.
    \item[C.]  
\end{enumerate}

% Page 236
\subsubsection*{Problem 3.12}
\rm{Replace} \tt{idivq} \rm{to} \tt{divq} \rm{}in line 5.

% Page 240
\subsubsection*{Problem 3.13}
\begin{tabular}{ccc}
    Conditions&data\_t&COMP\\\hline
    A.&int&<\\
    B.&short&>=\\
    C.&unsigned char&<=\\
    D.&long&!=\\
    &unsigned long&!=
\end{tabular}

% Page 241
\subsubsection*{Problem 3.14}
\begin{tabular}{ccc}
    Conditions&data\_t&TEST\\\hline
    A.&long&>=\\
    B.&short&==\\
    &unsigned short&==\\
    C.&unsigned char&>\\
    D.&int&<=
\end{tabular}

% Page 245
\subsubsection*{Problem 3.15}
\begin{enumerate}
    \item[A.] 4003fe
    \item[B.] 400425
    \item[C.] 400547 400545
    \item[D.] 400560
\end{enumerate}

% Page 248
\subsubsection*{Problem 3.16}
\begin{enumerate}
    \item[A.]
\begin{lstlisting}
void goto_cond(short a, short *p) {
    if (a == 0) {
        goto ret;
    }
    else {
        if (a <= *p) {
            goto ret;
        }
        *p = a;
    }
ret:
    return;
}
\end{lstlisting} 
    \item[B.] Because the compile split the if statement into to comparisons: (1) if a equals 0. (2) if *p < a. 
\end{enumerate}

\subsubsection*{Problem 3.17}
\begin{enumerate}
    \item[A.]
\begin{lstlisting}
long gotodiff_se_alt(long x, long y) 
{ 
    long result; 
    if (x < y) 
        goto x_lt_y; 
    ge_cnt++;
    result = x - y;
    return result;
x_lt_y:
    lt_cnt++;
    result = y - x;
    return result;
}
\end{lstlisting} 
    \item[B.] The original rule is better when there is no else statement.
\end{enumerate}

% Page 249
\subsubsection*{Problem 3.18}
\begin{lstlisting}
short test(short x, short y, short z) {
    short val = z + y - x;
    if (z > 5) {
        if (y > 2) {
            val = x / z;
        }
        else {
            val = x / y;
        }
    } else if (z < 3) 
        val = z / y;
    return val;
}
\end{lstlisting}

% Page 250
\subsubsection*{Problem 3.19}
\begin{enumerate}
    \item [A.] Approximately 40 cycles.
    \item [B.] 65 cycles.
\end{enumerate}

% Page 255
\subsubsection*{Problem 3.20}
\begin{enumerate}
    \item [A.] / (integral divide).
    \item [B.] 
    \begin{enumerate}
        \item [leaq:] assign x+15 to \%rbx
        \item [testq:] test x
        \item [cmovns:] when x is nonnegative, move x to \%rbx
        \item [sarq:] now if x is negative, \%rbx = x+15, otherwise x itself, so arithmetic right shift 4 bits will compute correctly both negative and nonnegative integer division.
    \end{enumerate}
\end{enumerate}

\subsubsection*{Problem 3.21}
\begin{lstlisting}
short test(short x, short y) {
    short val = y + 12;
    if (x < 0) {
        if (x >= y) {
            val = x | y;
        } else {
            val = x * y;
        }
    } else if (y > 10) {
        val = x / y;
    }
    return val;
}
\end{lstlisting}

% Page 257
\subsubsection*{Problem 3.22}
\begin{enumerate}
    \item [A.] Computation overflows.
    \item [D.] No.
\end{enumerate}

% Page 258
\subsubsection*{Problem 3.23}
\begin{enumerate}
    \item [A.] \%rbx for x, \%rcx for y, \%rdx for n.
    \item [B.] By using instruction leaq on line 5.
\end{enumerate}

% Page 260
\subsubsection*{Problem 3.24}
\begin{lstlisting}
short loop_while(short a, short b) {
    short result = 0;
    while (a > b) {
        result = result + a * b;
        a = a - 1;
    }
    return result;
}
\end{lstlisting}

% Page 262
\subsubsection*{Problem 3.25}
\begin{lstlisting}
long loop_while2(long a, long b) {
    long result = b;
    while (b > 0) {
        result = result * a;
        b = b - a;
    }
    return result;
}
\end{lstlisting}

% Page 264
\subsubsection*{Problem 3.26}
\begin{enumerate}
    \item [A.] Jump-to-middle method.
    \item [B.] 
    \begin{lstlisting}
short test_one(unsigned short x) {
    short val = 1;
    while (x) {
        val ^= x;
        x >>= 1
    }
    return val & 0;
}
    \end{lstlisting}
\end{enumerate}

% Page 267
\subsubsection*{Problem 3.27}
\begin{lstlisting}
long fibonacci_gd_goto(long n) {
    long i = 2;
    long next, first = 0, second = 1;
    if (n <= 1) 
        goto done;
loop:
    next = first + second;
    first = second;
    second = next;
    i++;
    if (i <= n) 
        goto loop;
done:
    return n;
}
\end{lstlisting}

\subsubsection*{Problem 3.28}
\begin{enumerate}
    \item [A.]
\begin{lstlisting}
short test_two(unsigned short x) {
    short val = 0;
    short i;
    for (i = 1, val = 65; i; i++) {
        val += val;
        val |= (x & 1);
        x >>= 1;
    }
    return val;
}
\end{lstlisting}
\end{enumerate}

\subsubsection*{Problem 3.29}
\begin{enumerate}
    \item [A.] The update-expr i++ will be skip, too. In this code, variable i will always equal to 1 and will never be updated.
    \item [B.]
\begin{lstlisting}
long sum = 0;
long i;
for (i = 0; i < 10;) {
    if (i & 1)
        goto update;
    sum += i;
update:
    i++;
}
\end{lstlisting}
\end{enumerate}

% Page 272
\subsubsection*{Problem 3.30}
\begin{enumerate}
    \item [A.] From -2 to 6.
    \item [B.] When x is -1, 1, 3, 6.
\end{enumerate}

% Page 273
\subsubsection*{Problem 3.31}
\begin{lstlisting}
void switcher(long a, long b, long c, long *dest) {
    long val;
    switch(a) {
        case 5:
            c = b ^ 15;
        case 0:
            val = c+112;
            break;
        case 2:
        case 7:
            val = (c + b) << 2;
            break;
        case 4:
            val = a;
            break;
        default:
            val = b;
    }
    *dest = val;
}
\end{lstlisting}

% Page 280
\subsubsection*{Problem 3.32}
\begin{tabular}{ccc|ccccc|c}
    \hline
    \multicolumn{3}{c|}{Instruction}&\multicolumn{5}{c|}{State values(at beginning)}&\\
    \hline
    Label&PC&Instruction&\%rdi&\%rsi&\%rax&\%rsp&*\%rsp&Description\\
    \hline
    M1&0x400560&callq&10&-&-&0x7fffffffe820&-&Call first(10)\\
    F1&0x400548&lea&10&-&-&0x7fffffffe818&0x400565&Entry of first\\
    F2&0x40054c&sub&10&11&-&0x7fffffffe818&0x400565&Assign 11 to \%rsi\\
    F3&0x400550&callq&9&11&-&0x7fffffffe818&0x400565&Call last(9, 11)\\
    L1&0x400540&mov&9&11&-&0x7fffffffe810&0x400555&Entry of last\\
    L2&0x400543&imul&9&11&9&0x7fffffffe810&0x400555&Assign 9 to \%rax\\
    L3&0x400547&retq&9&11&99&0x7fffffffe810&0x400555&Return 99 from last\\
    F4&0x400555&retq&9&11&99&0x7fffffffe818&0x400565&Resume first and return 99 from first\\
    M2&0x400565&mov&9&11&99&0x7fffffffe820&-&Resume main
    
\end{tabular}

% Page 282
\subsubsection*{Problem 3.33}
\begin{lstlisting}
int a, char b, long *u, char *v
\end{lstlisting}
or 
\begin{lstlisting}
int b, char a, long *v, char *u
\end{lstlisting}

% Page 288
\subsubsection*{Problem 3.34}
\begin{enumerate}
    \item [A.] a0, a1, a2, a3, a4, a5 get stored in callee-saved registers.
    \item [B.] a6. a7.
    \item [C.] There is only 6 callee-saved registers, but procedure P needs to store 8 local variables.
\end{enumerate}

% Page 289
\subsubsection*{Problem 3.35}
\begin{enumerate}
    \item [A.] x.
    \item [B.]
\begin{lstlisting}
long rfun(unsigned long x) {
    if (x == 0) 
        return x;
    unsigned long nx = x >> 2;
    long rv = rfun(nx);
    return rv + x;
}
\end{lstlisting}
\end{enumerate}

% Page 292
\subsubsection*{Problem 3.36}
\begin{tabular}{ccccc}
    Array&Element size&Total size&Start address&Element i\\
    \hline
    P&4&20&$x_P$&$x_P+4*i$\\
    Q&2&4&$x_Q$&$x_Q+2*i$\\
    R&8&72&$x_R$&$x_R+8*i$\\
    S&8&80&$x_S$&$x_S+8*i$\\
    T&8&16&$x_T$&$x_T+8*i$\\
\end{tabular}

% Page 294
\subsubsection*{Problem 3.37}
\begin{tabular}{cccc}
    Expression&Type&Value&Assembly code\\\hline
    P[1]&short&M[$x_P$]&movb 2(\%rdx), \%ax\\
    P+3+i&pointer&$x_P+2(i+3)$&leaq 6(\%rdx, \%rcx, 2), \%rax\\
    P[i*6-5]&short&M[$x_P+6i-5$]&movb -10(\%rdx, \%rcx, 12), \%ax\\
    P[2]&short&M[$x_P+4$]&movb 4(\%rdx), \%ax\\
    P[i+2]&pointer&$x_P+4+2i$&leaq 4(\%rdx, \%rcx, 2), \%rax
\end{tabular}

% Page 295
\subsubsection*{Problem 3.38}
M = 5, N = 7

% Page 298
\subsubsection*{Problem 3.39}
skip

\subsubsection*{Problem 3.40}
\begin{lstlisting}
void fix_set_diag_opt(fix_matrix A, int val) {
    int *ptr = &A[0][0];
    int *end = &A[N][N];
    while (ptr != end) {
        *ptr = val;
        ptr += 4*(N+1);
    }
}
\end{lstlisting}

% Page 304
\subsubsection*{Problem 3.41}
\begin{enumerate}
    \item [A.] 
    \begin{enumerate}
        \item [p:] 0
        \item [s.x:] 8
        \item [s.y:] 10
        \item [next:] 12
    \end{enumerate}
    \item [B.] 20
    \item [C.]
    \begin{lstlisting}
void st_init(struct test *st) {
    st->s.y = st->s.x;
    st->p = &(st->s.y);
    st->next = st;
}
    \end{lstlisting}
\end{enumerate}

% Page 305
\subsubsection*{Problem 3.42}
\begin{enumerate}
    \item [A.]
    \begin{lstlisting}
short test(struct ACE *ptr) {
    short result = 1;
    while (ptr != NULL) {
        result *= ptr->val;
        ptr = ptr->p;
    }
    return result;
}
    \end{lstlisting}
    \item [B.] This structure implements link list and procedure test computes the multiplication of every element in this link list.
\end{enumerate}

% Page 308
\subsubsection*{Problem 3.43}
\begin{tabular}{ccc}
    $expr$&$type$&Code\\\hline
    up->t1.u&long&movq (\%rdi), \%rax\\
    && movq \%rax, (\%rsi)\\
    up->t1.v&short&movw 8(\%rdi), \%ax\\
    && movw \%ax, (\%rsi)\\
    \&up->t1.w&char *&addq \$10, \%rdi\\
    &&movq \%rdi, (\%rsi)\\
    up->t2.a&int *&movq \%rdi, (\%rsi)\\
    up->t2.a[up->t1.u]&int&movq (\%rdi), \%rax\\
    &&movl (\%rdi, \%rax, 4), \%eax\\
    &&movl \%eax, (\%rsi)\\
    $\ast$up->t2.p&char&movb 8(\%rdi), \%rax\\
    &&movb (\%rax), \%al\\
    &&movb \%al, (\%rsi)
    
\end{tabular}

% Page 311
\subsubsection*{Problem 3.44}
\begin{tabular}{ccc}
    struct&offsets&alignment requirement\\\hline
    P1&0, 4, 8, 16&multiples of 8\\
    P2&0, 8, 16, 24&multiples of 8\\
    P3&0, 16&multiples of 8\\
    P4&0, 16&multiples of 8\\
    P5&0, 64&multiples of 8
\end{tabular}

\subsubsection*{Problem 3.45}
\begin{enumerate}
    \item [A.] 0, 8, 12, 14, 16, 24, 32, 40
    \item [B.] 48
    \item [C.] Optimized struct declaration:
    \begin{lstlisting}
struct {
    int *a;     // 0
    long e;     // 8
    double f;   // 16
    char *h;    // 24
    int g;      // 32
    float b;    // 36
    short d;    // 40
    char c;     // 42
} rec;
    \end{lstlisting}
    total size: 43 bytes.
\end{enumerate}

% Page 318
\begin{enumerate}
    \item [A.]
    \begin{tabular}{|cccccccc|c|}
        \hline
        00&00&00&00&00&40&00&76&Return address\\
        \hline
        01&23&45&67&89&AB&CD&EF&Saved \%rbx\\
        \hline
        \multicolumn{8}{|c|}{Unknown}&\\
        \hline
        \multicolumn{8}{|c|}{Unknown}&\%rsp pointed\\
        \hline
    \end{tabular}
    \item [B.]
    \begin{tabular}{|cccccccc|c|}
        \hline
        00&00&00&00&00&40&00&34&Corrupted return address\\
        \hline
        33&32&31&30&39&38&37&36&Corrupted saved \%rbx\\
        \hline
        35&34&33&32&31&30&39&38&written by gets\\
        \hline
        37&36&35&34&33&32&31&30&\%rsp pointed\\
        \hline
    \end{tabular}
    \item [C.] 0x400734
    \item [D.] \%rbx
    \item [E.] malloc should have strlen(buf+1) as its argument. And the code for get\_line doesn't check if the returned value is NULL.
\end{enumerate}

% Page 322
\subsubsection*{Problem 3.47}
\begin{enumerate}
    \item [A.] $2^{13}$
    \item [B.] $2^6=64$
\end{enumerate}





\end{document}